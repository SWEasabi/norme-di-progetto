\chapter{Standard ISO/IEC 12207:1997}\label{appendice-iso}
ISO/IEC 12207 è uno standard internazionale che definisce tutti i processi necessari per lo sviluppo e la manutenzione di sistemi software.

Per ogni processo esso stabilisce inoltre le attività da svolgere ed i risultati da produrre.

I processi sono divisi in tre categorie:
\begin{itemize}
    \item Processi primari;
    \item Processi di supporto;
    \item Processi organizzativi.
\end{itemize}

\section{Processi primari}

\subsection{Acquisizione}
Il processo descrive le attività dell'acquirente. Esso ha inizio con la necessità di un sistema, un prodotto software o un servizio.

Questo processo si conclude con la consegna del prodotto software e si compone delle seguenti attività:
\begin{itemize}
    \item Iniziazione;
    \item Preparazione della richiesta di proposta;
    \item Preparazione e aggiornamento del contratto;
    \item Monitoraggio dei fornitori;
    \item Accettazione e completamento.
\end{itemize}

\subsection{Fornitura}

Il processo descrive le attività del fornitore. Esso può avere inizio in due occasioni differenti:

\begin{itemize}
    \item Decisione di presentare una proposta in risposta alla richiesta del acquirente;
    \item Al momento della firma del contratto per la fornitura di un sistema, prodotto software o servizio.
\end{itemize}

Questo processo si conclude con la consegna del prodotto software e si compone delle seguenti attività:

\begin{itemize}
    \item Iniziazione;
    \item Preparazione di una risposta;
    \item Contratto;
    \item Pianificazione;
    \item Esecuzione e controllo;
    \item Revisione e valutazione;
    \item Consegna e completamento.
\end{itemize}

\subsection{Sviluppo}

Il processo descrive le attività dello sviluppatore. Esso norma lo sviluppo di un sistema software che risponda alle necessità del cliente.

Questo processo si conclude con la consegna del prodotto software e si compone delle seguenti attività:

\begin{itemize}
    \item Iniziazione;
    \item Analisi dei requisiti;
    \item Progettazione architetturale;
    \item Progettazione di dettaglio;
    \item Codifica e testing;
    \item Integrazione del software;
    \item Collaudo del software;
    \item Integrazione del sistema:
    \item Collaudo del sistema;
    \item Installazione.
\end{itemize}

\subsection{Operation}
Il processo si svolge in contemporanea alla faseG di Manutenzione. Ha lo scopo di mantenere operativo il sistema e di fornire supporto agli utenti.

Si compone delle seguenti attività:
\begin{itemize}
    \item Implementazione del processo;
    \item Operational testing;
    \item System operation;
    \item Supporto all'utente.
\end{itemize}

\subsection{Manutenzione}

Il processo descrive le attività del manutentore. Esso ha inizio quando il prodotto software viene sottoposto a modifiche del codice o della documentazione associata a causa di difetti o miglioramenti.

Questo processo si conclude con il ritiro del prodotto software.

Si compone delle seguenti attività:
\begin{itemize}
    \item Implementazione del processo;
    \item Analisi del problema o della modifica;
    \item Implementazione della modifica;
    \item Revisione/accettazione della manutenzione;
    \item Migrazione;
    \item Ritiro del software.
\end{itemize}


\section{Processi di supporto}

I processi di supporto hanno lo scopo di aiutare tutti gli altri processi attivi. Essi possono essere attivati da un processo primario o da un altro processo di supporto.

\subsection{Documentazione}
Il processo ha lo scopo di registrare le informazioni prodotte da un'attività o da un processo del ciclo di vita di un prodotto software.

Si compone delle seguenti attività:

\begin{itemize}
    \item Implementazione del processo;
    \item Identificazione dei documenti;
    \item Progettazione e sviluppo;
    \item Produzione;
    \item Manutenzione.
\end{itemize}

\subsection{Gestione della configurazione}

Il processo ha lo scopo di definire e mantenere l'integrità di tutti i componenti della configurazione e di renderli accessibili a chi ne ha diritto.

Si compone delle seguenti attività:
\begin{itemize}
    \item Implementazione del processo;
    \item Identificazione della configurazione;
    \item Controllo della configurazione;
    \item Registrazione dello stato della configurazione;
    \item Valutazione della configurazione;
    \item Gestione del rilascio e distribuzione.
\end{itemize}

\subsection{Gestione e controllo della qualità}

Il processo ha lo scopo di assicurare che tutti i prodotti e i processi del ciclo di vita siano conformi con gli standard ed i requisiti definiti.

Si compone delle seguenti attività:
\begin{itemize}
    \item Implementazione del processo;
    \item Accertamento del prodotto;
    \item Accertamento del processo;
    \item Accertamento dei sistemi di qualità.
\end{itemize}

\subsection{Verifica}

Il processo ha lo scopo di determinare se i prodotti software di un'attività soddisfano i requisiti o le condizioni imposte dalle attività precedenti. Il processo di Verifica dovrebbe essere integrato nei processi di Sviluppo, Fornitura e Manutenzione il prima possibile.

Si compone delle seguenti attività:

\begin{itemize}
    \item Implementazione del processo;
    \item Verifica
\end{itemize}

\subsection{Validazione}

Il processo ha lo scopo di accertarsi che il prodotto software o il sistema rispettino tutti i requisiti individuati.

Si compone delle seguenti attività:

\begin{itemize}
    \item Implementazione del processo;
    \item Validazione
\end{itemize}

\subsection{Revisione della congiunta}
Il processo ha lo scopo di revisionare lo stato ed i prodotti delle attività rispetto agli obiettivi definiti negli accordi, sia a livello di progetto che a livello tecnico. Tali revisioni avvengono durante l'intero ciclo di vita. Ad esse partecipano le parti interessate: solo il team se si revisiona un singolo componente o con la partecipazione del committente per revisioni sull'intero prodotto.

Si compone delle seguenti attività:
\begin{itemize}
    \item Implementazione del processo;
    \item Revisione della gestione del progetto;
    \item Revisioni tecniche.
\end{itemize}


\subsection{Audit}

Il processo ha lo scopo di determinare se vi è conformità dei prodotti e dei processi rispetto a requisiti, pianificazione e accordi. L'attività di auditing è svolta da personale che non ha partecipato direttamente allo sviluppo dei prodotti, servizi o sistemi oggetto delle revisioni.

Si compone delle seguenti attività:
\begin{itemize}
    \item Implementazione del processo;
    \item Audit.
\end{itemize}

\subsection{Risoluzione dei problemi}

Il processo ha lo scopo di analizzare e risolvere i problemi che vengono individuati in modo responsabile e documentato.

Si compone delle seguenti attività:
\begin{itemize}
    \item Implementazione del processo;
    \item Risoluzione del problema.
\end{itemize}

\section{Processi organizzativi}

I processi organizzativi sono di responsabilità dell'organizzazione che li attiva. Essi hanno lo scopo distabilire e implementare una struttura e gestire i processi del ciclo di vita e del personale. Inoltre sono utili per la gestione del continuo miglioramento dei processi e della struttura stessa.

\subsubsection{Gestione}

Il processo contiene le seguenti attività generiche per la gestione dei processi:

\begin{itemize}
    \item Implementazione del processo;
    \item Pianificazione
    \item Esecuzione e controllo;
    \item Revisione e valutazione;
    \item Terminazione.
\end{itemize}

\subsection{Infrastrutture}
Il processo ha lo scopo di istituire e manutenere le infrastrutture necessarie a qualsiasi altro processo.

L'infrastruttura può comprendere: hardware, software, strumenti, tecniche, standard e funzionalità per sviluppo, operazioni o manutenzione.

Si compone delle seguenti attività:
\begin{itemize}
    \item Implementazione del processo;
    \item Istituzione dell'infrastruttura;
    \item Manutenzione dell'infrastruttura.
\end{itemize}

\subsection{Miglioramento}

Il processo ha lo scopo di istituire, valutare, misurare, controllare e migliorare un processo di ciclo di vita del software.
Si compone delle seguenti attività:
\begin{itemize}
    \item Istituzione del processo;
    \item Valutazione del processo;
    \item Miglioramento del processo.
\end{itemize}

\subsection{Formazione}

Il processo ha lo scopo di fornire e mantenere personale istruito. L'acquisizione, la fornitura, lo sviluppo e molti altri processi dipendono da personale esperto e competente.

Si compone delle seguenti attività:
\begin{itemize}
    \item Implementazione del processo;
    \item Sviluppo del materiale per la formazione;
    \item Implementazione del piano di formazione.
\end{itemize}
