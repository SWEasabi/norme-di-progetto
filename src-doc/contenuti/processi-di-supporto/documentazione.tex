\section{Documentazione}\label{s:documentazione}

\subsection{Scopo}

Scopo del processo di documentazione è la raccolta di tutto quanto sia utile per rendere consistente, coeso e ripetibile il processo di produzione.

\subsection{Ciclo di vita dei documenti}

Ogni documento segue le fasi seguenti nel suo ciclo di vita. Inoltre queste fasi si ripeteranno per tutto il tempo in cui anche il prodotto software sarà vivo.

Questo ciclo di vita sarà incrementale e tracciato dalle versioni.

Prodotto software e documentazione coesistono per consentire il raggiungimento dello scopo di questo processo.

\paragraph{Approvazione delle modifiche} A risultato delle riunioni interne il responsabile approva le modifiche e quindi anche le aggiunte che dovranno essere eseguite al documento in oggetto.

\paragraph{Redazione} Un documento viene redatto da uno o più redattori. Il documento durante la redazione verrà incrementato di tutto quanto deciso durante le riunioni. Il documento viene considerato redatto quando se ne è completata la stesura.

\paragraph{Verifica} Quando un documento è stato redatto viene verificato dalla figura del \textit{verificatore}, questo ne controllerà consistenza, validità e ortografia.

\paragraph{Approvazione e Rilascio} Quando il documento sarà stato verificato il responsabile approverà e rilascerà il documento seguendo le regole di versionamento definite.

\subsubsection{Ciclo di vita dei verbali}

I verbali seguono lo stesso ciclo di vita dei docuementi, ma non avranno fasi di ripetizione, in quanto verranno redatti, verificati e rilasciati solamente una volta. Un verbale non sarà infatti un documento modificabile.

\subsection{Classificazione dei documenti}

I documenti vengono raggruppati in due classi principali definite dallo scopo delle stesse.

\subsubsection{Documenti ad uso interno}

I documenti ad uso interno sono destinati solo all'uso da parte dei membri del gruppo 

\begin{itemize}
    \item Verbali interni;
    \item Norme di progetto.
\end{itemize}

\subsubsection{Documenti ad uso esterno}

\begin{itemize}
    \item Verbali esterni;
    \item Analisi dei requisiti;
    \item Piano di progetto;
    \item Piano di qualifica.
\end{itemize}

\subsection{Nome dei file codificato}

Tutti i documenti seguono le regole generali nella definizione del nome del file. Su queste regole possono esserci poi specifiche espansioni o eccezioni.

\subsubsection{Regole generali}
\paragraph{Notazione}I nomi dei file seguono la notazione \textit{Kebab case}.

\[[\text{titolo-del-documento}]\]

\paragraph{Per il VCS} Per consentire il funzionamento delle pipeline di verifica\footnote{Vedasi sezione \ref{s:verifica}} dei documenti, nel sistema di \textit{VCS} i documenti in tex vanno nominati come segue.

\[\text{d}\_[\text{titolo-del-documento}]\]

\subsubsection{Regole specifiche per i verbali}
I nomi file dei verbali vengono codificati secondo la seguente struttura:

\[[TTT]\_[AAMMGG]\_[N]\]

Con:
\begin{itemize}
    \item \textbf{TTT}: tipologia di verbale come da tabella \Ref{table:t_verbali} ;
    \item \textbf{AAMMGG}: data di svolgimento dell'incontro a cui fa riferimento il verbale con ordinazione anno mese giorno;
    \item \textbf{N}: numero del verbale con partenza da 1 da inserire nel caso siano stati fatti più incontri nella medesima giornata.
\end{itemize}

\begin{table}[h]
    \centering
    \begin{tabularx}{0.7\linewidth}{X | l}
        \textbf{Tipologia verbale} & \textbf{Prefisso}\\
        \hline
        Verbale interno & VIN \\
        Verbali esterni & VEX\\
        Verbali di sprint & SPR \\
    \end{tabularx}
    \caption{Tipologie verbali}
    \label{table:t_verbali}
\end{table}



\subsection{Sigle e abbreviazioni}

All'interno dei documenti per brevità e semplicità di lettura verranno spesso utilizzate sigle e abbreviazioni, di seguito quelle con porrtata generale.

\begin{center}
    \begin{tabularx}{\linewidth}{l | X }            
        \textbf{Sigla} & \textbf{Definizione}\\
        \hline
        \textbf{AdR} & Analisi dei requisiti\\
        \textbf{NdP}& Norme di progetto\\
        \textbf{PdP}& Piano di progetto\\
        \textbf{PdQ}& Piano di qualifica\\
        \textbf{MU}& Manuale Utente\\
        \textbf{MM}& Manauele Manutentore\\
    \end{tabularx}
\end{center}

\subsection{Ciclo di vita dei documenti}

\subsection{Struttura}

Ogni documento ufficiale prodotto deve seguire una struttura precisa e definita per garantire consistenza comunicativa.

\begin{center}
    \begin{tabularx}{\linewidth}{l | X }            
        \textbf{Sezione} & \textbf{Scopo}\\
        \hline
        Frontespizio/copertina & Definisce per il lettore le informazioni base sul documento e su chi lo ha redatto\\
        Registro delle modifiche & Sono presenti tutti i cambiamenti al documento\\
        Tavola dei contenuti & Contiene l'indice dei contenuti presenti nel documento \\
        Glossario & Raccoglie le definizioni di tutti i termini di riguardo specifici per il documento\\
        Corpo del documento & Sono presenti tutte le informazioni comunicative del documento\\
    \end{tabularx}
\end{center}

\subsection{Stili testuali}
\subsection{Elementi grafici}
\subsection{Strumenti di redazione e tecnologici}

\paragraph{LaTeX} Utilizzato per stendere i documenti principali poiché consente di utilizzare metodologie strutturate di versionamento\footnote{Vedasi sottosezione \Ref{sss:vcs} per ulteriori informazioni.} e di verifica dei contenuti redatti.

\paragraph{GitHubActions} Per controllare la consistenza del documento è presente un'azione che controlla la compilazione del documento in LaTeX.
