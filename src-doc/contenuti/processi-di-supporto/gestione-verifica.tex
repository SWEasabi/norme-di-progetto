\section{Gestione della verifica}\label{s:verifica}
\subsection{Scopo}

Scopo del processo di gestione della verifica è individuare errori all'interno del prodotto durante la sua produzione e la sua manutenzione.

Sia la documentazione\footnote{Vedasi sezione \ref{s:documentazione} per approfondire.} che il software sono soggetti a questo processo.

\subsection{Premesse fondamentali}

La documentazione e il software seguono due processi differenti per la verifica ma con delle metodologie e degli standard simili.

In ogni caso, la verifica segue il concetto del \textit{fail fast}, si vanno quindi ad eseguire verifiche dal più piccolo al più grande aglomenrato, sia di codice che di testo.

Più in generale il processo di verifica ha due forme:
\begin{itemize}
    \item \textbf{Analisi statica} che non richiede l'esecuzione dell'oggetto di verifica.
    \item \textbf{Analisi dinamica} che si può effettuare solamente durante l'esecuzione dell'oggetto di verifica.
\end{itemize}

\paragraph{Documentazione} Poiché la documentazione è redatta tramite l'utilizzo di \textit{LaTeX} si rende necessario testarne la compilazione ad ogni nuova modifica.

\subsection{Metodologie e ciclo di vita}

Poiché per il versionamento viene utilizzato GitHub il processo di verifica è stato fatto collimare con il percorso di verifica che propone lo strumento.

Il percorso di verifica segue le seguenti fasi:

\begin{enumerate}
    \item viene aperta una issue relativamente alla parte di oggetto di verifica che deve essere espansa;
    \item viene aperto un branch relativo alla issue o al gruppo di issue se le stesse sono in relazione tra loro;
    \item completato il ciclo di modifiche viene aperta una \textit{pull request} che richiedeal verificatore di analizzare le modifiche fatte;
    \item il sistema automatizzato tramite \textit{GitHubActions} effettua un' analisi statica;
    \item il verificatore controlla che quanto è stato scritto corrisponda con quanto affermato durante la stesura delle issue e controlla che i test vadano a buon fine;
    \item la modifica viene aggiunta all'oggetto principale e nuovamente verificata staticamente.
\end{enumerate}

\subsection{Testing}

Ogni manufatto prodotto viene testato il più possibile in maniera automatica.

\paragraph{Verifica della compilazione} Si testa che il sorgente, sia esso documentale che di software compili correttamente. In particolare a livello documentale è importante che il sorgente compili anche quando questo viene integrato con il resto del documento.

\paragraph{Test di unità}

Ogni modulo aggiunto o modificato al sorgente viene testato singolarmente per pregiudicarne anomalie o errori al suo interno.

\paragraph{Test di integrazione}

Ogni modulo viene unito agli altri e se ne testa la sua integrazione con i moduli verso i quali si interfaccia.

\paragraph{Test di sistema} Si verifica che il sistema faccia quello per cui è stato pensato e che rispetti quanto trovato nell'analisi dei requisiti.

\subsection{Strumenti}

\subsubsection{Verifiche automatiche alla alla documentazione}

Alla documentazione si applica una verifica della compilazione e poi su questo vengono effettuate delle verifiche manuali da parte di un verificatore, il quale controllerà validità di quanto redatto e con questo anche l'assenza di eventuali errori grammaticali o di sintassi che possano eventualmente essere presenti.