\chapter{Processi di supporto}
\section{Documentazione}

\subsection{Classificazione dei documenti}

I documenti vengono raggruppati in due classi principali definite dallo scopo delle stesse.

\subsubsection{Documenti ad uso interno}

I documenti ad uso interno sono destinati solo all'uso da parte dei membri del gruppo 

\begin{itemize}
    \item Verbali interni;
    \item Norme di progetto.
\end{itemize}

\subsubsection{Documenti ad uso esterno}

\begin{itemize}
    \item Verbali esterni;
    \item Analisi dei requisiti;
    \item Piano di progetto;
    \item Piano di qualifica.
\end{itemize}

\subsection{Nome dei file codificato}

Tutti i documenti seguono le regole generali nella definizione del nome del file. Su queste regole possono esserci poi specifiche espansioni o eccezioni.

\subsubsection{Regole generali}
\paragraph{Notazione}I nomi dei file seguono la notazione \textit{Kebab case}.

\paragraph{Indicatore di versione}Ogni file deve descrivere di che versione del documento sta parlando.



\subsubsection{Verbali}
I nomi file dei verbali vengono codificati secondo la seguente struttura:

\[[TTT]\_[AAMMGG]\_[N]\]

Con:
\begin{itemize}
    \item \textbf{TTT}: tipologia di verbale come da tabella \Ref{table:t_verbali} ;
    \item \textbf{AAMMGG}: data di svolgimento dell'incontro a cui fa riferimento il verbale con ordinazione anno mese giorno;
    \item \textbf{N}: Numero del verbale con partenza da 1 da inserire nel caso siano stati fatti più incontri nella medesima giornata.
\end{itemize}

\begin{table}[h]
    \centering
    \begin{tabularx}{0.7\linewidth}{X | l}
        \textbf{Tipologia verbale} & \textbf{Prefisso}\\
        \hline
        Verbale interno & VIN \\
        Verbali esterni & VEX\\
        Verbali di sprint & SPR \\
    \end{tabularx}
    \caption{Tipologie verbali}
    \label{table:t_verbali}
\end{table}



\subsection{Sigle e abbreviazioni}

All'interno dei documenti per brevità e semplicità di lettura verranno spesso utilizzate sigle e abbreviazioni, di seguito quelle con porrtata generale.

\begin{center}
    \begin{tabularx}{\linewidth}{l | X }            
        \textbf{Sigla} & \textbf{Definizione}\\
        \hline
        \textbf{AdR} & Analisi dei requisiti\\
        \textbf{NdP}& Norme di progetto\\
        \textbf{PdP}& Piano di progetto\\
        \textbf{PdQ}& Piano di qualifica\\
        \textbf{MU}& Manuale Utente\\
        \textbf{MM}& Manauele Manutentore\\
    \end{tabularx}
\end{center}

\subsection{Ciclo di vita dei documenti}

\subsection{Struttura}

Ogni documento ufficiale prodotto deve seguire una struttura precisa e definita per garantire consistenza comunicativa.

\begin{center}
    \begin{tabularx}{\linewidth}{l | X }            
        \textbf{Sezione} & \textbf{Scopo}\\
        \hline
        Frontespizio/copertina & Definisce per il lettore le informazioni base sul documento e su chi lo ha redatto\\
        Registro delle modifiche & Sono presenti tutti i cambiamenti al documento\\
        Tavola dei contenuti & Contiene l'indice dei contenuti presenti nel documento \\
        Glossario & Raccoglie le definizioni di tutti i termini di riguardo specifici per il documento\\
        Corpo del documento & Sono presenti tutte le informazioni comunicative del documento\\
    \end{tabularx}
\end{center}

\subsection{Stili testuali}
\subsection{Elementi grafici}
\subsection{Strumenti di redazione e tecnologici}

\paragraph{LaTeX} Utilizzato per stendere i documenti principali poiché consente di utilizzare metodologie strutturate di versionamento\footnote{Vedasi sottosezione \Ref{sss:vcs} per ulteriori informazioni.} e di verifica dei contenuti redatti.

\paragraph{GitHubActions} Per controllare la consistenza del documento è presente un'azione che controlla la compilazione del documento in LaTeX.

\section{Gestione di configurazione}

Per la configurazione del prodotto e per la gestione documentale vengono utilizzate varie strategie per organizzare e rendere consistente.

\subsection{Strumenti tecnologici}

\subsubsection{VCS}\label{sss:vcs} Per il versionamento e tracciamento delle modifiche apportate ai documenti e al codice steso durante il progetto viene utilizzato il \textit{version control system} Git. 

In abbinata a Git viene utilizzato GitHub per permettere facile collaborazione oltre ad accorpare in un unico luogo anche gli strumenti di ITS e CI/CD pipeline.

\subsubsection{ITS} Per il tracciamento delle issue viene utilizzato GitHub.

\subsection{Strutturazione dei repository}


\section{Gestione della qualità}
\section{Gestione della verifica}
\section{Validazione}