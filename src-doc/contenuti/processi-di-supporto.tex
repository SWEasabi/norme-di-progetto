\chapter{Processi di supporto}

\section{Documentazione}

\subsection{Classificazione dei documenti}

I documenti vengono raggruppati in due classi principali definite dallo scopo delle stesse.

\begin{itemize}
    \item Documenti ad uso interno
    \item Documenti ad uso esterno
\end{itemize}

\subsection{Nome dei file e versionamento}



I verbali vengono codificati secondo la seguente struttura:

\begin{center}
    \begin{tabularx}{\linewidth}{l | l | X}
        \textbf{Tipologia verbale} & \textbf{Prefisso} & \textbf{Numerazione}\\
        \hline
        Verbale interno & VIN & X* \\
        Verbale esterno & VEX & X* \\
    \end{tabularx}
\end{center}

\subsection{Sigle e abbreviazioni}

All'interno dei documenti per brevità e semplicità di lettura verranno spesso utilizzate sigle e abbreviazioni, di seguito quelle con porrtata generale.

\begin{center}
    \begin{tabularx}{\linewidth}{l | X }            
        \textbf{Sigla} & \textbf{Definizione}\\
        \hline
        \textbf{AdR} & Analisi dei requisiti\\
        \textbf{NdP}& Norme di progetto\\
        \textbf{PdP}& Piano di progetto\\
        \textbf{PdQ}& Piano di qualifica\\
        \textbf{MU}& Manuale Utente\\
        \textbf{MM}& Manauele Manutentore\\
    \end{tabularx}
\end{center}

\subsection{Ciclo di vita dei documenti}

\subsection{Struttura}

Ogni documento ufficiale prodotto deve seguire una struttura precisa e definita per garantire consistenza comunicativa.

\begin{center}
    \begin{tabularx}{\linewidth}{l | X }            
        \textbf{Sezione} & \textbf{Scopo}\\
        \hline
        Frontespizio/copertina & Definisce per il lettore\\
        Registro delle modifiche & VEX\\
        Tavola dei contenuti \\
        Glossario &\\
        Corpo del documento &\\
    \end{tabularx}
\end{center}

\subsection{Stili testuali}
\subsection{Elementi grafici}
\subsection{Strumenti di stesura}



\section{Gestione di configurazione}

\subsection{Strumenti tecnologici}
\subsection{Strutturazione dei repository}


\section{Gestione della qualità}
\section{Gestione della verifica}
\section{Validazione}