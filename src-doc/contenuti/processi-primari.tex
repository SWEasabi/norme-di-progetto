\chapter{Processi primari}

\section{Fornitura}
\subsection{Scopo}
In questa sezione vengono descritti i documenti che compongono il processo di fornitura, la loro struttura e gli strumenti utilizzati.

\subsection{Descrizione}
Il processo di fornitura ha lo scopo di consegnare al richiedente un prodotto che soddisfi le richieste concordate con il fornitore.
Queste richieste verranno analizzate in uno studio di fattibilità, che servirà per definire costi e tempi del progetto. Concordato ciò il proponente e il fornitore stipuleranno un contratto per la consegna del

\subsection{Aspettative}
Durante l’intero processo di fornitura, il gruppo intende:
\begin{itemize}
    \item Mantenere una struttura chiara dei documenti;
    \item Avere un constante dialogo con il proponente;
    \item Stabilire vincoli e tempistiche per la realizzazione del prodotto;
    \item Effettuare una continua verifica del lavoro.
\end{itemize}

\subsection{Attività}
Il processo di fornitura si compone delle seguenti attività:
\begin{itemize}
    \item Inizializzazione;
    \item Contrattazione delle richieste;
    \item Pianificazione;
    \item Esecuzione e controllo;
    \item Revisione e valutazione;
    \item Rilascio.
\end{itemize}

\subsection{Strumenti}
Per questo processo verranno utilizzati i seguenti strumenti:

\section{Sviluppo}

\subsection{Scopo}

\subsection{Aspettative}

\subsection{Attività}