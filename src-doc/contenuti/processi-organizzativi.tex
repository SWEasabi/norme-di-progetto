\chapter{Processi organizzativi}
\subsection{Gestione Organizzativa}

\subsubsection{Scopo}

In questa sezione saranno illustrate le tecniche di coordinamento utilizzate dal gruppo e gli obiettivi del processo, i quali comprendono:

\begin{itemize}
\item Definire le responsabilità e le competenze di ciascun membro dell'organizzazione.
\item Sviluppare un modello organizzativo che favorisca la collaborazione tra i membri dell'organizzazione e l'efficienza delle attività svolte.
\item Pianificare il lavoro in modo da rispettare le scadenze e utilizzare al meglio le risorse disponibili.
\item Monitorare costantemente l'andamento delle attività per individuare eventuali problemi e adottare le correzioni necessarie.
\item Promuovere la comunicazione interna all'organizzazione e la condivisione di informazioni e conoscenze tra i membri.
\item Migliorare continuamente i processi organizzativi e le performance dell'organizzazione nel lungo termine.
\end{itemize}

\subsubsection{Aspettative}

Illustriamo ora le seguenti attese rispetto al processo:

\begin{itemize}
\item Utilizzare processi efficaci per regolare le attività e renderle più efficienti dal punto di vista economico.
\item Ottenere una stima accurata dei tempi, delle risorse e dei costi necessari per svolgere le attività.
\item Garantire un'adeguata esecuzione e controllo delle attività svolte.
\item Revisionare e valutare periodicamente le attività svolte per migliorare continuamente i processi organizzativi e le performance dell'organizzazione.
\item Favorire una cultura della collaborazione e della condivisione delle conoscenze tra i membri dell'organizzazione.
\end{itemize}

\subsubsection{Descrizione}

Le attività di gestione sono:

\begin{itemize}
\item Assegnazione dei ruoli e dei compiti ai membri del gruppo.
\item Definizione degli obiettivi e dello scopo del progetto.
\item Implementazione dei processi organizzativi necessari per svolgere le attività.
\item Pianificazione e stima dei tempi, delle risorse e dei costi necessari per svolgere le attività, attraverso la creazione di un piano di lavoro ben definito.
\item Esecuzione delle attività previste, seguendo il piano di lavoro definito in precedenza.
\item Controllo costante dell'andamento delle attività svolte, attraverso la valutazione periodica e l'eventuale adozione di misure correttive.
\item Revisione e valutazione periodica delle attività svolte per identificare, in caso, inefficienze o opportunità di miglioramento.
\item Garanzia della comunicazione tra i membri del gruppo, attraverso la definizione di canali di comunicazione efficaci e la promozione di una cultura della collaborazione e della condivisione delle conoscenze.
\end{itemize}

\section{Ruoli}

\subsection{Responsabile di progetto}
Il responsabile di progetto ha il compito di coordinare e dirigere il gruppo di lavoro, assicurandosi che ogni membro abbia chiari i propri compiti e che siano rispettati i tempi e le scadenze. In particolare, i compiti assegnati al responsabile di progetto sono:

\begin{itemize}
\item Definizione dei requisiti del progetto
\item Pianificazione e monitoraggio dell'intero progetto
\item Gestione delle risorse e delle scadenze
\item Comunicazione con il cliente
\end{itemize}

\subsection{Amministratore di progetto}
L'amministratore di progetto ha il compito di gestire gli aspetti amministrativi e organizzativi del progetto, fornendo supporto al responsabile di progetto e agli altri membri del gruppo. In particolare, i compiti assegnati all'amministratore di progetto sono:

\begin{itemize}
\item Gestione della documentazione relativa al progetto, garantendo l'archiviazione e la gestione delle informazioni in modo corretto e coerente tramite versionamento e configurazione.
\item Istanziare le infrastrutture atte al supporto.
\item Gestione delle problematiche relative alla gestione dei processi.
\end{itemize}

\subsection{Analista}
L'analista di progetto ha il compito di analizzare i requisiti e le esigenze del cliente e di definire le specifiche tecniche del progetto. In particolare, i compiti assegnati all'analista di progetto sono:

\begin{itemize}
\item Raccolta dei requisiti del cliente, tramite interviste, analisi dei documenti e discussioni con gli stakeholder del progetto.
\item Definizione delle specifiche funzionali e tecniche del progetto, in accordo con il committente e gli altri membri del gruppo.
\item Analisi dei rischi e delle problematiche del progetto, definendo strategie e soluzioni per mitigarli.
\item Redazione della documentazione tecnica relativa al progetto, come ad esempio i diagrammi dei casi d'uso, i diagrammi delle classi.
\end{itemize}

\subsection{Progettista}
L'obiettivo principale del progettista è quello di definire e pianificare l'architettura del progetto, ne segue il suo sviluppo ma non la sua manutenzione. Le sue attività includono:

\begin{itemize}
\item Definizione dell'architettura del sistema e delle sue componenti;
\item Definizione delle specifiche tecniche per la realizzazione del sistema;
\item Verifica delle soluzioni proposte in termini di fattibilità e coerenza con i requisiti.
\end{itemize}

\subsection{Ruoli}

\subsubsection{Programmatore}

Il programmatore è responsabile della scrittura del codice del progetto software. Le sue attività includono:

\begin{itemize}
\item Implementazione delle specifiche tecniche definite dal progettista;
\item Implementazione di componenti di supporto per attuare verifica e validazione;
\item Collaborazione con il progettista per risolvere eventuali problematiche riscontrate nell'implementazione del sistema.
\end{itemize}

\subsubsection{Verificatore}

Il verificatore è incaricato lungo tutta l'attività di progetto di lavorare in stretta collaborazione con gli altri membri del team per verificare la qualità del prodotto. Le sue attività includono:

\begin{itemize}
\item Verificare il prodotto tramite gli strumenti e le procedure definite nelle Norme di Progetto.
\item Collaborare con l'autore notificando errori presenti nel prodotto.
\item Verifica dell'adeguatezza dei requisiti e delle specifiche tecniche.
\end{itemize}

\subsection{Formazione}

La formazione interna dei singoli membri del gruppo viene attuata tramite la pratica di auto-formazione, nella modalità di studio autonomo delle tecnologie proposte dal proponente nella fase di presentazione del capitolato e gli incontri dedicati.

\subsubsection{Materiale utilizzato}

Si identificano i seguenti materiali utilizzati:

\begin{itemize}
\item LaTeX
\item GitHub
\item AngularJS
\item Java Spring
\item Git
\end{itemize}

\subsection{Procedure}

Il gruppo adotta le seguenti procedure per la sua organizzazione interna e il flusso comunicativo esterno, si adottano le procedure seguenti:

\subsection{Gestione delle comunicazioni}

\section{Comunicazioni}

\subsection{Comunicazione interna}

Le comunicazioni interne del gruppo si attuano tramite i canali di comunicazione che provvedono gli applicativi Discord e Telegram. Questi applicativi permettono di creare un ambiente di lavoro virtuale e a distanza.

\subsection{Comunicazione esterna}

Le comunicazioni verso il proponente sono gestite dal responsabile di progetto e avvengono nelle seguenti modalità:

\begin{itemize}
\item tramite posta elettronica, usando l'indirizzo e-mail del gruppo sweasabi@gmail.com
\item tramite la piattaforma Meet, per colloqui diretti con Imola Informatica.
\item tramite Telegram, utilizzando un gruppo dedicato tra "sweasabi" e Lorenzo Patera (Imola Informatica)
\end{itemize}

\section{Gestione degli incontri}

\subsection{Incontri interni}

\subsubsection{Verbali di riunioni interne}

Al termine di ogni incontro interno, viene compilato un verbale dalla figura di segretario. Viene descritto all'interno del documento un riassunto dell'incontro, approvato poi dal responsabile.

\subsection{Incontri esterni}

Il responsabile di progetto progetta un incontro esterno con il proponente. Il proponente oppure il Committente possono richiedere incontri con il gruppo di progetto, successivamente il responsabile propone una data a calendario ed un orario.

\subsubsection{Verbali di riunioni esterne}

Al termine dell'incontro esterno, il segretario ha il compito di redigere il verbale esterno, contenente il riassunto dell'incontro, approvato poi dal responsabile.

\section{Gestione dei rischi}

È compito del responsabile identificare e rendere noti i rischi. Si adotta la seguente procedura di gestione dei rischi notificati nel piano di progetto:

\begin{enumerate}
\item Identificazione di nuovi rischi e controllo di quelli già noti.
\item Rimodulazione della procedura di gestione dei rischi al bisogno.
\end{enumerate}

\subsection{Codifica dei rischi}

Si utilizza la seguente codifica per i rischi identificati:

\begin{itemize}
\item \textbf{RT:} Rischi Tecnologici.
\item \textbf{RO:} Rischi Organizzativi.
\item \textbf{RI:} Rischi Interpersonali.
\end{itemize}

\section{Strumenti}

Il gruppo utilizza i seguenti strumenti per la gestione del progetto:

\begin{itemize}
\item \textbf{Git}
\item \textbf{GitHub}
\item \textbf{Google Drive}
\item \textbf{Telegram}
\item \textbf{Discord}
\item \textbf{Meet}
\end{itemize}